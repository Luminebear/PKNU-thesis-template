%%%%%%%%%%%%%%%%%%%%%%%%%%%%%%%%%%%%%%%%%%%%%%%%%%%%%%%%%%%%%%%%%%%%%%%%%%%%%%%%%%%%%%%%%%%%%%%%%%%%
%
% 							PKNU Graduate thesis template in LaTeX format
%
%%%%%%%%%%%%%%%%%%%%%%%%%%%%%%%%%%%%%%%%%%%%%%%%%%%%%%%%%%%%%%%%%%%%%%%%%%%%%%%%%%%%%%%%%%%%%%%%%%%%



% % documentclass: 새로 정의한 PKNU-thesis class를 사용
% %% PKNU-thesis는 세가지 옵션을 제공합니다.
% %% doctor: 박사과정 | master: 석사과정
% %% korean: 한글논문 | english: 영문논문
% %% final: 최종판    | draft: 시험판
% %% pdfdoc : 선택하지 않으면 북마크와 colorlink를 만들지 않습니다.(Generate bookmark and colorlink if enabled)
\documentclass[master,english,final,pdfdoc]{PKNU-thesis}



% % usepackage: 클래스 기본 패키지로 geometry, indentfirst, color
% %% 필요에따라 \usepackage 명령어를 이용하여 패키지를 이용하면 됩니다.
\usepackage{amsmath}
\usepackage{amssymb}
\usepackage{enumitem}
\usepackage{algorithm}
\usepackage{algpseudocode}
\usepackage{comment}
\usepackage{physics}
\usepackage{anyfontsize}
\usepackage{lipsum}
\newcommand{\argmax}{\operatornamewithlimits{arg\,max}}

% user custom package
\usepackage[style=phys, biblabel=brackets, block=space]{biblatex}
\usepackage{tocloft}
\usepackage{listings}

\renewcommand\cftchapaftersnum{.}
\renewcommand\cftsecaftersnum{.}
\renewcommand\cftsubsecaftersnum{.}
\renewcommand\cftchapnumwidth{2.0em}
\renewcommand\cftsecnumwidth{1.4em}
\renewcommand\cftsecindent{2em}
\renewcommand\cftsubsecnumwidth{2.4em}
\renewcommand\cftsubsecindent{3.4em}

\renewcommand{\thechapter}{\Roman{chapter}}
\renewcommand{\thesection}{\arabic{section}}
\renewcommand{\theequation}{\arabic{chapter}.\arabic{equation}}



% % title 논문 제목(title of thesis)
% %% options [default: (none)]
% %% - korean: 한글제목(korean title) | english: 영문제목(english title)
\title[korean]{다람쥐 헌 쳇바퀴에 타고파}
\title[english]{The Quick Brown Fox Jumps Over The Lazy Dog}

% @note 표지에 출력되는 제목을 강제로 줄바꿈하려면 \linebreak 을 삽입.
%       \\ 나 \newline 등을 사용하면 안됩니다. (아래는 예시)
%
% If you want to begin a new line in cover, use \linebreak .
% See examples above.



% % author 저자 이름
% %% parameters family_name, given_name 성, 이름을 구분해서 입력
% %% options [default: (none)]
% - korean: 한글이름 | chinese: 한문이름 | english: 영문이름
% ex) \author[english]{family name in english}{given name in english}
%
% If you are a foreigner (this means you have no korean name),
% You must fill the korean name as blank, instead of deleting it or commenting it out.
% \author[korean]{}{}
% \author[english]{Donald}{Trump}
\author[korean]{홍}{길 동}
\author[english]{Hong}{Gil Dong}



% % advisor 지도교수 이름 (복수가능)
% @usage   \advisor[options]{...한글이름...}{...영문이름...}{signed|nosign}
% @options [default: major동동
% - major: 주 지도교수  | coopr: 공동 지도교수
\advisor[major]{김 민 수}{Minsoo Kim}{nosign}
% \advisor[coopr]{옥 영 향}{Younghyang Ok}{nosign}
%
% 지도교수 한글이름은 입력하지 않아도 됩니다.
% You may not input advisor's korean name
% like this \advisor[major]{}{Chang, Kee Joo}{signed}
%

% @command department {학과이름}{학위종류} - 아래 표에 따라 코드를 입력
% @command department {department code}{degree field}
%
% department code table
% 학위모집요강에 따른 순서를 따름
% NS    |   간호학과          |   Department of Nursing
% SC    |   과학컴퓨팅학과    |   Department of Scientific Computing
% PH    |   물리학과          |   Department of Physics 
% AM    |   응용수학과        |   Department of Applied Mathematics
% ST    |   통계학과          |   Department of Statistics
% CH    |   화학과            |   Department of Chemistry
%
% science: 이학 | engineering: 공학 | business : 경영학
% 박사논문의 경우는 학위종류를 입력하지 않아도 됩니다.
% If you write Ph.D. dissertation, you cannot input degree field.

\department{SC}{science}

% @command studentid 학번(ID)
\studentid{}

% @command referee 심사위원 (석사과정 3인, 박사과정 5인)
\referee[1]{홍 상 직}
\referee[2]{옥 영 향}
\referee[3]{홍 상 직}
\referee[4]{옥 영 향}
\referee[5]{홍 상 직}
% Of course english name is available

% @command refsign 심사위원 서명
\refsign[1]{./figure/Erwin_Schrödinger_signature.png}
\refsign[2]{./figure/Erwin_Schrödinger_signature.png}
\refsign[3]{./figure/Erwin_Schrödinger_signature.png}
\refsign[4]{./figure/Erwin_Schrödinger_signature.png}
\refsign[5]{./figure/Erwin_Schrödinger_signature.png}

% @command approvaldate 학위논문승인일
% @param   year,month,day 연,월,일 순으로 입력
\approvaldate{2023}{02}{28}

% @command refereedate 심사위원논문심사일
% @param   year,month,day 연,월,일 순으로 입력
\refereedate{2024}{02}{28}

\bibliography{thesis}
% 본문 시작
\begin{document}
    % 앞표지, 속표지, 학위논문 제출승인서, 학위논문 심사완료 검인서는
    % 클래스 옵션을 final로 지정해주면 자동으로 생성되며,
    % 반대로 옵션을 draft로 지정해주면 생성되지 않습니다.

    % 목차 (Table of Contents) 생성
    % \tableofcontents
    % 그림목차 (List of Figures) 생성
    % \listoffigures
    % 표목차 (List of Tables) 생성
    % \listoftables

    % 위의 세 종류의 목차는 한꺼번에 다음 명령으로 생성할 수도 있습니다.
    \makecontents
    % 초록 (abstract)
    % 초록은 한국어논문의 경우 한국어 -> 영어 순으로, 영어논문의 경우 영어 -> 한국어로 생성됩니다.
    % 작성은 논문 종류와 관계없이 반드시 영어, 한국어 순으로 작성해주세요!
    \begin{abstract}
        {\lipsum[1]}
        {누구든지 체포 또는 구속을 당한 때에는 적부의 심사를 법원에 청구할 권리를 가진다. 
        위원은 정당에 가입하거나 정치에 관여할 수 없다. 대통령은 국민의 보통·평등·직접·비밀선거에 의하여 선출한다. 
        정부는 회계연도마다 예산안을 편성하여 회계연도 개시 90일전까지 국회에 제출하고, 
        국회는 회계연도 개시 30일전까지 이를 의결하여야 한다.

        공개하지 아니한 회의내용의 공표에 관하여는 법률이 정하는 바에 의한다. 
        모든 국민은 인간으로서의 존엄과 가치를 가지며, 행복을 추구할 권리를 가진다. 
        국가는 개인이 가지는 불가침의 기본적 인권을 확인하고 이를 보장할 의무를 진다. 
        국군의 조직과 편성은 법률로 정한다. 국민경제자문회의의 조직·직무범위 기타 필요한 사항은 법률로 정한다.}
    \end{abstract}

%% 이하의 본문은 LaTeX 표준 클래스 report 양식에 준하여 작성하시면 됩니다.
%% 하지만 part는 사용하지 못하도록 제거하였으므로, chapter가 문서 내의
%% 최상위 분류 단위가 됩니다.
%% You cannot use 'part'

	\chapter{Introduction}
% \section[서론]{\hyperlink{toc}{서론}}
%%%%% PLEASE KEEP THIS LOCATION %%%%%%
\pagenumbering{arabic}
\setcounter{page}{1}
%%%%% PLEASE KEEP THIS LOCATION %%%%%%
\section{Purpose of research}
\label{subsec:intro}
Write your contents here.

\section{Definitions of terms}
본 연구에서 사용하는 용어에 대한 정의와 각 용어의 배경이 되는 이론을 간략히 정리하면 다음과 같다.

	\newpage
\chapter{Background}
% \section[데이터 및 방법론]{\hyperlink{toc}{데이터 및 방법론}}
\label{dm}

\setcounter{figure}{0}
\setcounter{table}{0}
\setcounter{equation}{0}

Add your contents here.

\section{Data}

\subsection{Data 1}
% \subsection[발의안 데이터]{\hyperlink{toc}{발의안 데이터}}

\subsection{Data 2}
% \subsection[잠재적으로 공동발의에 영향을 끼치는 요인]{\hyperlink{toc}{잠재적으로 공동발의에 영향을 끼치는 요인}}

\section{Methods}
% \subsection[방법론]{\hyperlink{toc}{방법론}}

\subsection{Method 1}
% \subsubsection[주성분 분석(PCA)]{\hyperlink{toc}{주성분 분석(PCA)}}

\subsection{Method 2}

\subsection{Method 3}
% \subsubsection[측정값]{\hyperlink{toc}{측정값}}


	\input{./dat/03.dataandmethod}
	\newpage
\chapter{Results}
% \section[결과]{\hyperlink{toc}{결과}}
\label{chap:res}

\setcounter{figure}{0}
\setcounter{table}{0}
\setcounter{equation}{0}

\section{Results A}

\section{Results A.1}

\subsection{Results A.2}

\subsection{Results A.3}

\newpage
\section{Results B}

\section{Analysis for results A}

\section{Analysis for results B}

	\newpage %\newpage\ \newpage
\chapter{Conclusion and Discussion}
% \section[결론]{\hyperlink{toc}{결론}}
\newpage

	\appendix

\setcounter{figure}{0}
\setcounter{table}{0}
\setcounter{equation}{0}

\chapter{example appendix}
\label{appendix:A}
	% \input{./dat/07.reference}

%%
%% 참고문헌 시작
%% References
%%

% For BibLaTeX
\addcontentsline{toc}{chapter}{Bibliography}
\printbibliography

% For BibTeX / natbib
% \let\Section\section
% \def\section*#1{\Section*{{\hyperlink{toc}{#1}}}}
% \renewcommand{\thesection}{}
% \bibliographystyle{chicago}
% \bibliography{thesis}
% \nocite{apsrev42Control}
% \bibliographystyle{./revtex4-2_bst/apsrev4-2}
% \bibliography{thesis}
% Copy and Paste the code in .bib file below
% @CONTROL{REVTEX42Control}
% @CONTROL{apsrev42Control,author="00",editor="1",pages="1",title="0",year="0"}

%%
%% 감사의 글 시작
%% Acknowledgement
%%
% @command acknowledgement 감사의글
% @options [default: 클래스 옵션 korean|english]
% - korean : 한글타이틀 | english : 영문타이틀
    \acknowledgement[4]
    Don't be afraid to give up the good to go for the great. - John D. Rockfeller
%% 본문 끝
\end{document}