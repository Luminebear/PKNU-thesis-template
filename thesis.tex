%%%%%%%%%%%%%%%%%%%%%%%%%%%%%%%%%%%%%%%%%%%%%%%%%%%%%%%%%%%%%%%%%%%%%%%%%%%%%%%%%%%%%%%%%%%%%%%%%%%%
%
% 							PKNU Graduate thesis template in LaTeX format
%
%%%%%%%%%%%%%%%%%%%%%%%%%%%%%%%%%%%%%%%%%%%%%%%%%%%%%%%%%%%%%%%%%%%%%%%%%%%%%%%%%%%%%%%%%%%%%%%%%%%%
%%%%%%%%%%%%%%%%%%%%%%%%%%%%%%%%%%%%%%%%%%%%%%%%%%
% DOCUMENTCLASS: PKNU-thesis class
%%%%%%%%%%%%%%%%%%%%%%%%%%%%%%%%%%%%%%%%%%%%%%%%%%
% %% Options
% %%%% doctor: 박사과정 | master:  석사과정
% %%%% korean: 한글논문 | english: 영문논문
% %%%% final:  최종판   | draft:   시험판
% %%%% pdfdoc: 선택하지 않으면 북마크와 colorlink를 만들지 않습니다.(Generate bookmark and colorlink if enabled)
\documentclass[doctor,english,final,pdfdoc]{PKNU-thesis}



%%%%%%%%%%%%%%%%%%%%%%%%%%%%%%%%%%%%%%%%%%%%%%%%%%
% USEPACKAGE
%%%%%%%%%%%%%%%%%%%%%%%%%%%%%%%%%%%%%%%%%%%%%%%%%%
% Default package by PKNU-thesis class: geometry, indentfirst, color
% 필요에따라 \usepackage 명령어로 패키지를 이용하면 됩니다.
\usepackage{amsmath}
\usepackage{amssymb}
\usepackage{enumitem}
\usepackage{algorithm}
\usepackage{algpseudocode}
\usepackage{comment}
\usepackage{physics}
\usepackage{anyfontsize}
\usepackage{lipsum}
\newcommand{\argmax}{\operatornamewithlimits{arg\,max}}

% user custom package
\usepackage[style=phys, biblabel=brackets, block=space]{biblatex}
\usepackage{tocloft}
\usepackage{listings}

% TOC settings
\renewcommand\cftchapaftersnum{.}
\renewcommand\cftsecaftersnum{.}
\renewcommand\cftsubsecaftersnum{.}
\renewcommand\cftchapnumwidth{2.0em}
\renewcommand\cftsecnumwidth{1.4em}
\renewcommand\cftsecindent{2.0em}
\renewcommand\cftsubsecnumwidth{2.4em}
\renewcommand\cftsubsecindent{3.4em}

\renewcommand{\thechapter}{\Roman{chapter}}
\renewcommand{\thesection}{\arabic{section}}
\renewcommand{\theequation}{\arabic{chapter}.\arabic{equation}}



%%%%%%%%%%%%%%%%%%%%%%%%%%%%%%%%%%%%%%%%%%%%%%%%%%
% TITLE OF THESIS 논문제목
%%%%%%%%%%%%%%%%%%%%%%%%%%%%%%%%%%%%%%%%%%%%%%%%%%
% %% Options [default: (none)]
% %%%% korean: 한글제목(korean title) | english: 영문제목(english title)
%
% 표지에 출력되는 제목을 강제로 줄바꿈하려면 \linebreak 을 삽입.
% \\ 나 \newline 등을 사용하면 안됩니다. (아래는 예시)
% If you want to begin a new line in cover, use \linebreak .
\title[korean]{다람쥐 헌 쳇바퀴에 타고파}
\title[english]{The Quick Brown Fox Jumps Over The Lazy Dog}



%%%%%%%%%%%%%%%%%%%%%%%%%%%%%%%%%%%%%%%%%%%%%%%%%%
% AUTHOR 저자명
%%%%%%%%%%%%%%%%%%%%%%%%%%%%%%%%%%%%%%%%%%%%%%%%%%
% %% Options [default: (none)]
% %%%% korean: 한글이름 | chinese: 한문이름 | english: 영문이름
% %% Parameters 
% %%%% family_name, given_name 성, 이름을 구분해서 입력
% ex) \author[english]{family name in english}{given name in english}
%
% If you are a foreigner (this means you have no korean name),
% You must fill the korean name as blank, instead of deleting it or commenting it out.
\author[korean]{홍}{길 동}
\author[english]{Hong}{Gil Dong}



%%%%%%%%%%%%%%%%%%%%%%%%%%%%%%%%%%%%%%%%%%%%%%%%%%
% ADVISOR 지도교수
%%%%%%%%%%%%%%%%%%%%%%%%%%%%%%%%%%%%%%%%%%%%%%%%%%
% %% Options [default: major]
% %%%% major: 주 지도교수  | coopr: 공동 지도교수
% %% Parameters
% %%%% 한글이름(in Korean), 영문이름(in English), signed | nosign 
% ex) \advisor[options]{...한글이름...}{...영문이름...}{signed|nosign}
\advisor[major]{김 민 수}{Minsoo Kim}{nosign}
% \advisor[coopr]{옥 영 향}{Younghyang Ok}{nosign}
%
% 지도교수 한글이름은 입력하지 않아도 됩니다.
% You may not input advisor's korean name
% ex) \advisor[major]{}{Chang, Kee Joo}{signed}



%%%%%%%%%%%%%%%%%%%%%%%%%%%%%%%%%%%%%%%%%%%%%%%%%%
% DEPARTMENT 학과명
%%%%%%%%%%%%%%%%%%%%%%%%%%%%%%%%%%%%%%%%%%%%%%%%%%
% %% Parameters
% %%%% department code(학과코드), degree field(학위종류)
% %%%% 학과코드는 아래 표에 따라 코드를 입력
% %% Department code table
% 학위모집요강에 따른 순서를 따름
%% Natural Science
% NS    |   간호학과          |   Department of Nursing
% SC    |   과학컴퓨팅학과    |   Department of Scientific Computing
% PH    |   물리학과          |   Department of Physics 
% AM    |   응용수학과        |   Department of Applied Mathematics
% ST    |   통계학과          |   Department of Statistics
% CH    |   화학과            |   Department of Chemistry
%% Engineering
% IT    |   IT융합응용공학과  |   Department of IT Convergence & Application Engineering
% CM    |   컴퓨터공학과      |   Department of Computer Engineering
% MC    |   기계공학과        |   Department of Mechanical Engineering
% RA    |   냉동공조공학과    |   Department of Refrigeration and Air-conditioning Engineering
% EC    |   전기공학과        |   Department of Electrical Engineering
% IT    |   전자공학과        |   Department of Electronic Engineering
% %% Options for degree field
% science: 이학 | engineering: 공학 | business : 경영학
% 박사논문의 경우는 학위종류를 입력하지 않아도 됩니다.
% If you write Ph.D. dissertation, you cannot input degree field.
\department{PH}{science}



%%%%%%%%%%%%%%%%%%%%%%%%%%%%%%%%%%%%%%%%%%%%%%%%%%
% STUDENT ID
%%%%%%%%%%%%%%%%%%%%%%%%%%%%%%%%%%%%%%%%%%%%%%%%%%
% 비워있어도 무관
\studentid{}



%%%%%%%%%%%%%%%%%%%%%%%%%%%%%%%%%%%%%%%%%%%%%%%%%%
% REFREEE 심사위원
%%%%%%%%%%%%%%%%%%%%%%%%%%%%%%%%%%%%%%%%%%%%%%%%%%
% %% Options [enumeration of refrees]
% %%%% Up to 5
% %% Parameters
% %%%% 한글이름(in Korean) or 영문이름(in English)
% %% 석사과정 3인, 박사과정 5인
% %% 국문논문 작성시 한글로, 영문논문 작성시 영문이름 기입
% Of course english name is available
\referee[1]{홍 상 직}
\referee[2]{옥 영 향}
\referee[3]{홍 상 직}
\referee[4]{옥 영 향}
\referee[5]{홍 상 직}



%%%%%%%%%%%%%%%%%%%%%%%%%%%%%%%%%%%%%%%%%%%%%%%%%%
% SIGNATURE OF REFREES 심사위원 서명
%%%%%%%%%%%%%%%%%%%%%%%%%%%%%%%%%%%%%%%%%%%%%%%%%%
% %% Options [enumeration of refrees]
% %%%% Up to 5
% %% Parameters
% %%%% Path of a signature figure
\refsign[1]{./figure/Erwin_Schrödinger_signature.png}
\refsign[2]{./figure/Erwin_Schrödinger_signature.png}
\refsign[3]{./figure/Erwin_Schrödinger_signature.png}
\refsign[4]{./figure/Erwin_Schrödinger_signature.png}
\refsign[5]{./figure/Erwin_Schrödinger_signature.png}



%%%%%%%%%%%%%%%%%%%%%%%%%%%%%%%%%%%%%%%%%%%%%%%%%%
% APPROVAL DATE 학위논문 승인일
%%%%%%%%%%%%%%%%%%%%%%%%%%%%%%%%%%%%%%%%%%%%%%%%%%
% %% Parameters
% %%%% Year, Month, Day 연,월,일 순으로 입력
\approvaldate{2024}{02}{28}



%%%%%%%%%%%%%%%%%%%%%%%%%%%%%%%%%%%%%%%%%%%%%%%%%%
% REFREEE DATE 심사위원 논문심사일
%%%%%%%%%%%%%%%%%%%%%%%%%%%%%%%%%%%%%%%%%%%%%%%%%%
% %% Parameters
% %%%% Year, Month, Day 연,월,일 순으로 입력
\refereedate{2024}{02}{28}



%%%%%%%%%%%%%%%%%%%%%%%%%%%%%%%%%%%%%%%%%%%%%%%%%%
% BIBLIOGRAPHY 참고문헌 (By BibLaTeX)
%%%%%%%%%%%%%%%%%%%%%%%%%%%%%%%%%%%%%%%%%%%%%%%%%%
\bibliography{thesis}



%%%%%%%%%%%%%%%%%%%%%%%%%%%%%%%%%%%%%%%%%%%%%%%%%%
% BEGINNING OF THE DOCUMENT 문서 시작
%%%%%%%%%%%%%%%%%%%%%%%%%%%%%%%%%%%%%%%%%%%%%%%%%%
\begin{document} 
    %%%%%%%%%%%%%%%%%%%%%%%%%%%%%%%%%%%%%%%%%%%%%%%%%%
    % Main Cover, Inner Cover, Approval of Statement 앞표지, 속표지, 학위논문 인준서
    %%%%%%%%%%%%%%%%%%%%%%%%%%%%%%%%%%%%%%%%%%%%%%%%%%
    % PKNU-thesis 클래스 옵션을 final로 지정해주면 자동으로 생성, draft로 지정해주면 생성되지 않습니다.

    %%%%%%%%%%%%%%%%%%%%%%%%%%%%%%%%%%%%%%%%%%%%%%%%%%
    % Table of Contents, List of Figures, List of Tables 목차, 그림목차, 표목차
    %%%%%%%%%%%%%%%%%%%%%%%%%%%%%%%%%%%%%%%%%%%%%%%%%%
    % Use \tableofcontents, \listoffigures, \listoftables
    % Or use \makecontents for showing together.
    \makecontents

    %%%%%%%%%%%%%%%%%%%%%%%%%%%%%%%%%%%%%%%%%%%%%%%%%%
    % ABSTRACT 논문요약
    %%%%%%%%%%%%%%%%%%%%%%%%%%%%%%%%%%%%%%%%%%%%%%%%%%
    % %% Parameters
    % %%%% Abstract in English, Abstract in Korean
    % 초록은 한국어논문의 경우 한국어 -> 영어 순으로, 영어논문의 경우 영어 -> 한국어로 생성됩니다.
    % 작성은 논문 종류와 관계없이 반드시 영어, 한국어 순으로 작성해주세요!
    \begin{abstract}
        {\lipsum[1]}
        {
            피고인의 자백이 고문·폭행·협박·구속의 부당한 장기화 또는 기망 기타의 방법에 의하여 자의로 진술된 것이 아니라고 인정될 때 또는 정식재판에 있어서 피고인의 자백이 그에게 불리한 유일한 증거일 때에는 이를 유죄의 증거로 삼거나 이를 이유로 처벌할 수 없다.
            대통령의 국법상 행위는 문서로써 하며, 이 문서에는 국무총리와 관계 국무위원이 부서한다. 
            군사에 관한 것도 또한 같다. 
            연소자의 근로는 특별한 보호를 받는다. 
            감사원은 원장을 포함한 5인 이상 11인 이하의 감사위원으로 구성한다. 
            모든 국민은 고문을 받지 아니하며, 형사상 자기에게 불리한 진술을 강요당하지 아니한다. 
            국가는 법률이 정하는 바에 의하여 재외국민을 보호할 의무를 진다.
        }
    \end{abstract}

    %%%%%%%%%%%%%%%%%%%%%%%%%%%%%%%%%%%%%%%%%%%%%%%%%%
    % Main Body 본문
    %%%%%%%%%%%%%%%%%%%%%%%%%%%%%%%%%%%%%%%%%%%%%%%%%%
    %% 이하의 본문은 LaTeX 표준 클래스 report 양식에 준하여 작성하시면 됩니다.

	\chapter{Introduction}
% \section[서론]{\hyperlink{toc}{서론}}
%%%%% PLEASE KEEP THIS LOCATION %%%%%%
\pagenumbering{arabic}
\setcounter{page}{1}
%%%%% PLEASE KEEP THIS LOCATION %%%%%%
\section{Purpose of research}
\label{subsec:intro}
Write your contents here.

\section{Definitions of terms}
본 연구에서 사용하는 용어에 대한 정의와 각 용어의 배경이 되는 이론을 간략히 정리하면 다음과 같다.

	\newpage
\chapter{Background}
% \section[데이터 및 방법론]{\hyperlink{toc}{데이터 및 방법론}}
\label{dm}

\setcounter{figure}{0}
\setcounter{table}{0}
\setcounter{equation}{0}

Add your contents here.

\section{Data}

\subsection{Data 1}
% \subsection[발의안 데이터]{\hyperlink{toc}{발의안 데이터}}

\subsection{Data 2}
% \subsection[잠재적으로 공동발의에 영향을 끼치는 요인]{\hyperlink{toc}{잠재적으로 공동발의에 영향을 끼치는 요인}}

\section{Methods}
% \subsection[방법론]{\hyperlink{toc}{방법론}}

\subsection{Method 1}
% \subsubsection[주성분 분석(PCA)]{\hyperlink{toc}{주성분 분석(PCA)}}

\subsection{Method 2}

\subsection{Method 3}
% \subsubsection[측정값]{\hyperlink{toc}{측정값}}


	\newpage
\chapter{Data and Methods}
% \section[데이터 및 방법론]{\hyperlink{toc}{데이터 및 방법론}}
\label{dm}

\setcounter{figure}{0}
\setcounter{table}{0}
\setcounter{equation}{0}

Add your contents here.

\section{Data}

\subsection{Data 1}
% \subsection[발의안 데이터]{\hyperlink{toc}{발의안 데이터}}

\subsection{Data 2}
% \subsection[잠재적으로 공동발의에 영향을 끼치는 요인]{\hyperlink{toc}{잠재적으로 공동발의에 영향을 끼치는 요인}}

\section{Methods}
% \subsection[방법론]{\hyperlink{toc}{방법론}}

\subsection{Method 1}
% \subsubsection[주성분 분석(PCA)]{\hyperlink{toc}{주성분 분석(PCA)}}

\subsection{Method 2}

\subsection{Method 3}
% \subsubsection[측정값]{\hyperlink{toc}{측정값}}


	\newpage
\chapter{Results}
% \section[결과]{\hyperlink{toc}{결과}}
\label{chap:res}

\setcounter{figure}{0}
\setcounter{table}{0}
\setcounter{equation}{0}

\section{Results A}

\section{Results A.1}

\subsection{Results A.2}

\subsection{Results A.3}

\newpage
\section{Results B}

\section{Analysis for results A}

\section{Analysis for results B}

	\newpage %\newpage\ \newpage
\chapter{Conclusion and Discussion}
% \section[결론]{\hyperlink{toc}{결론}}
\newpage

	\appendix

\setcounter{figure}{0}
\setcounter{table}{0}
\setcounter{equation}{0}

\chapter{example appendix}
\label{appendix:A}

    %%%%%%%%%%%%%%%%%%%%%%%%%%%%%%%%%%%%%%%%%%%%%%%%%%
    % BIBLIOGRAPHY 참고문헌
    %%%%%%%%%%%%%%%%%%%%%%%%%%%%%%%%%%%%%%%%%%%%%%%%%%
    % For BibLaTeX
    \addcontentsline{toc}{chapter}{Bibliography}
    \printbibliography

    % For BibTeX / natbib
    % \let\Section\section
    % \def\section*#1{\Section*{{\hyperlink{toc}{#1}}}}
    % \renewcommand{\thesection}{}
    % \bibliographystyle{chicago}
    % \bibliography{thesis}
    % \nocite{apsrev42Control}
    % \bibliographystyle{./revtex4-2_bst/apsrev4-2}
    % \bibliography{thesis}
    % Copy and Paste the code in .bib file below
    % @CONTROL{REVTEX42Control}
    % @CONTROL{apsrev42Control,author="00",editor="1",pages="1",title="0",year="0"}

    %%%%%%%%%%%%%%%%%%%%%%%%%%%%%%%%%%%%%%%%%%%%%%%%%%
    % ACKNOWLEDGEMENT 감사의 글
    %%%%%%%%%%%%%%%%%%%%%%%%%%%%%%%%%%%%%%%%%%%%%%%%%%
    % %% Options [enumeration of refrees]
    % %%%% 1: English thesis, and acknowledgement
    \acknowledgement[1]
    Don't be afraid to give up the good to go for the great. - John D. Rockfeller

%%%%%%%%%%%%%%%%%%%%%%%%%%%%%%%%%%%%%%%%%%%%%%%%%%
% END OF THE DOCUMENT 문서 끝
%%%%%%%%%%%%%%%%%%%%%%%%%%%%%%%%%%%%%%%%%%%%%%%%%%
\end{document}