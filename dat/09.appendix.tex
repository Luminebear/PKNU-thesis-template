%\begin{appendices}
% \appendix
% \renewcommand{\thesection}{부록 \Alph{section}.}
%\renewcommand{\theequation}{\Alph{section}.\arabic{equation}}
%\renewcommand{\thefigure}{\Alph{section}.\arabic{figure}}
%\def\adjbmk#1{\texorpdfstring{#1}{\Alph{section}:~~#1}}

%\setcounter{figure}{0}
%\setcounter{table}{0}
%\setcounter{equation}{0}

% \hsection{민주주의 지수와 갈등의 차원수}
%\hsection{\adjbmk{민주주의 지수와 갈등의 차원수}}
%그림~\ref{fig:pca_all}을 보면 시간이 지남에 따라 민주주의 지수는 증가하는데 반해 분산 비율은 점점 감소 즉, 차원수가 점점 증가함을 알 수 있다. 이에 대해 본 연구에서는 민주주의 지수($x(t)$)와 차원수($\epsilon(t)$) 사이의 관계를
%\begin{equation}
%    \dot{\epsilon} = x(t) \epsilon(t)
%\end{equation}
%로 가정하고, $x(t) \sim t$로 근사하여
%\begin{equation}
%    \epsilon(t) \sim e^{t^2}
%\end{equation}
%의 관계식을 얻었다. 이를 실제 데이터와 비교한 결과 그림~\ref{fig:ind_dim}을 얻었다.
%\begin{figure*}
%    \centering
%    \includegraphics[width=\textwidth]{figure/exp.pdf}
%    \caption{민주주의 지수에 따른 차원수. $e^{t^2}$(파란선)과 실제 데이터(빨간선)의 비교. 민주주의 지수가 낮은 상태와 높은 상태에서는 실제 데이터와 %일치하는 경향이 있지만, 과도기 상태에서는 정확하게 일치하지 않는다.}
%\label{fig:ind_dim}
%\end{figure*}
%실제 데이터의 경우 $e^{t^2}$과 비슷하게 맞추기 위해 재규격화(rescaling)를 했다. 그 결과 민주주의 지수가 낮을 때와 높을 때는 그래프와 일치하는 경향이 보이고 과도기적 상태에서는 정확히 일치하지는 않지만 차원수가 감소하는 경향을 나타냈다.
%\end{appendices}

%\newpage